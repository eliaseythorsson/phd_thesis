\documentclass[]{book}
\usepackage{lmodern}
\usepackage{amssymb,amsmath}
\usepackage{ifxetex,ifluatex}
\usepackage{fixltx2e} % provides \textsubscript
\ifnum 0\ifxetex 1\fi\ifluatex 1\fi=0 % if pdftex
  \usepackage[T1]{fontenc}
  \usepackage[utf8]{inputenc}
\else % if luatex or xelatex
  \ifxetex
    \usepackage{mathspec}
  \else
    \usepackage{fontspec}
  \fi
  \defaultfontfeatures{Ligatures=TeX,Scale=MatchLowercase}
\fi
% use upquote if available, for straight quotes in verbatim environments
\IfFileExists{upquote.sty}{\usepackage{upquote}}{}
% use microtype if available
\IfFileExists{microtype.sty}{%
\usepackage{microtype}
\UseMicrotypeSet[protrusion]{basicmath} % disable protrusion for tt fonts
}{}
\usepackage[margin=1in]{geometry}
\usepackage{hyperref}
\hypersetup{unicode=true,
            pdftitle={PhD thesis},
            pdfauthor={Elías Sæbjörn Eyþórsson},
            pdfborder={0 0 0},
            breaklinks=true}
\urlstyle{same}  % don't use monospace font for urls
\usepackage{natbib}
\bibliographystyle{apalike}
\usepackage{longtable,booktabs}
\usepackage{graphicx,grffile}
\makeatletter
\def\maxwidth{\ifdim\Gin@nat@width>\linewidth\linewidth\else\Gin@nat@width\fi}
\def\maxheight{\ifdim\Gin@nat@height>\textheight\textheight\else\Gin@nat@height\fi}
\makeatother
% Scale images if necessary, so that they will not overflow the page
% margins by default, and it is still possible to overwrite the defaults
% using explicit options in \includegraphics[width, height, ...]{}
\setkeys{Gin}{width=\maxwidth,height=\maxheight,keepaspectratio}
\IfFileExists{parskip.sty}{%
\usepackage{parskip}
}{% else
\setlength{\parindent}{0pt}
\setlength{\parskip}{6pt plus 2pt minus 1pt}
}
\setlength{\emergencystretch}{3em}  % prevent overfull lines
\providecommand{\tightlist}{%
  \setlength{\itemsep}{0pt}\setlength{\parskip}{0pt}}
\setcounter{secnumdepth}{5}
% Redefines (sub)paragraphs to behave more like sections
\ifx\paragraph\undefined\else
\let\oldparagraph\paragraph
\renewcommand{\paragraph}[1]{\oldparagraph{#1}\mbox{}}
\fi
\ifx\subparagraph\undefined\else
\let\oldsubparagraph\subparagraph
\renewcommand{\subparagraph}[1]{\oldsubparagraph{#1}\mbox{}}
\fi

%%% Use protect on footnotes to avoid problems with footnotes in titles
\let\rmarkdownfootnote\footnote%
\def\footnote{\protect\rmarkdownfootnote}

%%% Change title format to be more compact
\usepackage{titling}

% Create subtitle command for use in maketitle
\newcommand{\subtitle}[1]{
  \posttitle{
    \begin{center}\large#1\end{center}
    }
}

\setlength{\droptitle}{-2em}
  \title{PhD thesis}
  \pretitle{\vspace{\droptitle}\centering\huge}
  \posttitle{\par}
  \author{Elías Sæbjörn Eyþórsson}
  \preauthor{\centering\large\emph}
  \postauthor{\par}
  \predate{\centering\large\emph}
  \postdate{\par}
  \date{2018-09-25}

\usepackage{booktabs}
\usepackage{amsthm}
\makeatletter
\def\thm@space@setup{%
  \thm@preskip=8pt plus 2pt minus 4pt
  \thm@postskip=\thm@preskip
}
\makeatother

\usepackage{amsthm}
\newtheorem{theorem}{Theorem}[chapter]
\newtheorem{lemma}{Lemma}[chapter]
\theoremstyle{definition}
\newtheorem{definition}{Definition}[chapter]
\newtheorem{corollary}{Corollary}[chapter]
\newtheorem{proposition}{Proposition}[chapter]
\theoremstyle{definition}
\newtheorem{example}{Example}[chapter]
\theoremstyle{definition}
\newtheorem{exercise}{Exercise}[chapter]
\theoremstyle{remark}
\newtheorem*{remark}{Remark}
\newtheorem*{solution}{Solution}
\begin{document}
\maketitle

{
\setcounter{tocdepth}{1}
\tableofcontents
}
\chapter{Preamble}\label{preamble}

I am currently writing my PhD thesis on the impact of pneumococcal
vaccination in Iceland. I decided to host the thesis on github and
distribute on social media. I am doing this for mostly selfish reasons.
I believe I will be more motivated if my productiveness -- or lack
thereof, is held accountable to anyone who wishes to check. I would be
grateful for any and all comments on any aspect of the thesis under
construction.

\chapter{Introduction}\label{intro}

\emph{Streptococcus pneumoniae} is a commensal bacterium found in the
nasopharynx of humans where it plays an integral role in the normal
upper respiratory flora. It is also a common pathogen and is one of the
most common bacterial causes of disease in humans. In classical medical
texts pneumococcus is described as a Gram-positive lancet-shaped coccus
that is usually found in pairs. In fact, pneumococcus is \emph{the}
Gram-positive coccus, being first bacteria that Christian Gram noted
which retained the dark aniline-gentian violet stain that now bears his
name \citep{Gram1884}. Pneumococcus was first isolated in 1881 by two
microbiologist, George M. Sternberg in the United States and Louis
Pasteur in France \citep{Pasteur1881, Sternberg1881, Watson1993}. In
both cases the sample from which the bacterium was isolated originated
from healthy carriers, rather than ill patients. The causal association
between this newly discovered bacterium and pneumonia was firmly
established only five years later \citep{Weichselbaum1886}, and in the
following decade, all clinical presentations of pneumococcal infection
had been described \citep{Austrian1981}. Pneumococcus has gone by many
names since its first isolation in 1881. Originally it was named
\emph{Micrococcus pasteuri} by Sternberg \citep{Sternberg1881} but, by
1920 a scientific consensus was reached in which it was agreed that the
official name should be \emph{Diplococcus pneumoniae}
\citep{Winslow1920}. It was not until 1974 that pneumococcus received
its current name, \emph{Streptococcus pneumoniae} \citep{Deibel1974}.

Pneumococcus is encapsulated by a polysaccharide coating that protects
it from environmental factors. The polysaccharide capsule acts as an
``invisibility cloak'' to the human immune system, which is rendered
unable to detect pneumococcus except through certain patterns in the
oligosaccharides contained within the capsule \citep{Epstein1995}. Based
on these patterns, pneumococcus has been classified into over 97
different serotypes to date. Additionally, as the capsule contains only
polysaccharides and not proteins, the immune response is T-cell
independent and therefore poorly immunogenic, even after identification
by the immune system \citep{Geno2015b}. These characteristics are the
fundamental challenges faced by scientists engineering new pneumococcal
vaccines. The significance of serotypes on the development of vaccines
should not be understated. The failure of the original attempt of Wright
and colleagues in 1911 to prevent pneumococcal pneumonia with
vaccination was indeed due to lack of knowledge about serotype-specific
immunogenicity \citep{Wright1914}.

All serotypes of pneumococcus have the potential to cause disease in
humans. However, some are more virulent than others. The prevalence of
asymptomatic carriage in the nasopharynx varies greatly by serotype as
does the propensity of serotypes to cause clinical infections.
Quantifying the pathogenic potential of serotypes is difficult as both
their prevalence and propensity to cause disease need to be considered.
With few exceptions, acquisition of a new serotype into the
nasopharyngeal flora proceeds the onset of clinical disease caused by
that serotype. Pneumococcal epidemiology is dominated by this effect -
children act as a reservoir of asymptomatic pneumococcal carriage from
which other children and adults acquire serotypes that may lead to
symptomatic disease. Because of this, vaccinations that decrease the
pneumococcal carriage in children have the potential of reducing the
incidence of disease both in other unvaccinated children and in adults.
This phenomenon is called a herd-effect, and is integral to the
development of vaccination strategies to combat pneumococcal disease.

The infectious manifestations of pneumococcal disease are, broadly
speaking, local infections of the respiratory tract and infections of
previously sterile tissue. They range from common to uncommon, and from
benign to serious. The most common infectious manifestation of
pneumococcus in acute otitis media (AOM) -- an infection of the middle
ear. AOM is the most common reason for physician visit and for
antimicrobial prescription in the paediatric population. However, the
disease course is benign and rarely results in permanent disability. The
pathogenesis of AOM is complex and can both be caused by viral and
bacterial pathogens. Bacterial AOM is most often caused by pneumococcus
and \emph{Haemophilus influenzae}. Acute sinusitis is another common,
though benign, manifestation of pneumococcal disease. Pneumococcus is
also a common cause of pneumonia -- the disease from which it gets its
name. Pneumonia often requires hospitalization and intravenous
antimicrobial treatment, and, though uncommonly, can lead to permanent
disability and death. Finally, if pneumococcus gains access to normally
sterile tissue, it may cause invasive infections. This includes
bacteremia, an infection of the blood, and meningitis, an infection of
the meninges. These infectious manifestations are grouped together as
invasive pneumococcal disease (IPD). Whilst IPD is extremely uncommon,
the consequences can be disastrous. The case fatality ratio from
pneumococcal meningitis in Iceland is estimated at 15.3\%.

Pneumococcus became an early target for vaccine development because of
the broad range of disease caused by pneumococcus. The vaccine benefit
can be quantified in two different ways. On one hand it can prevent
uncommon but serious manifestations of disease and has the potential to
prevent death and serious disability. On the other hand, it also
prevents common infections that cumulatively present a large healthcare
burden due to frequent physician visits, antimicrobial prescriptions as
well as work hours lost by parents and caregivers. The earliest attempts
to use vaccination to lessen the morbidity and mortality associated with
pneumococcus date back to 1911 when Wright used whole cell innoculi to
vaccinated miners in South Africa \citep{Wright1914}. In the following
decades, multiple animal studies showed that injection of the
pneumococcal polysaccharide coating in animals protected against
subsequent pneumococcal infections. On the basis of these findings, the
first polysaccharide vaccine was shown to be effective
\citep{Macleod1945}. This lay the foundation for modern polysaccharide
pneumococcal vaccines. However, it soon became apparent that the
polysaccharide vaccines were not adequately immunogenic in young
children, the ill or the elderly. In response, pneumococcal conjugate
vaccines were developed. The first such vaccine to be mass produced was
the heptavalent Prevenar developed by Pfizer Pharmaceuticals.

Iceland is an independent island nation, isolated in the mid-Atlantic,
with a homogeneous population of roughly 330,000 individuals. The first
systematic program of vaccination against pneumococcus in Iceland began
in April 2011, when the 10-valent pneumococcal \emph{Haemophilus
influnzae} protein-D conjugate vaccine (Synflorix, PHiD-CV10) was
introduced into the national paediatric vaccination program. The vaccine
program entailed two primary doses given at three and five months of
age, and a booster dose at twelve months. No catch-up program was
undertaken. Prior to the introduction, no systematic vaccination effort
had been undertaken in Iceland. Iceland, as other Nordic countries, has
a rich legacy of national health-related registers. With a wealth of
medical documentation, a unique whole-population ecological study
examining the impact of systematic pneumococcal vaccination was enabled.

\section{\texorpdfstring{Clinical manifestations of \emph{Streptococcus
pneumoniae}}{Clinical manifestations of Streptococcus pneumoniae}}\label{clinical-manifestations-of-streptococcus-pneumoniae}

The relationship between pneumococcus and humans is complex. Most
children are colonized by pneumococcus within the first months of life.
The serotype distribution of the initial colonization is influenced by
the distribution of serotypes within the family. Over the course of the
child's lifetime, they will be colonized by many different serotypes.
The child's immune system will learn to recognize the currently
colonizing serotypes and will either clear the colonization or maintain
a equilibrium in which the serotype is kept within a certain limit of
reproduction. Consequently, the contribution of pneumococcus to the
human upper respiratory flora is in a state of constant flux. New
serotypes enter while old exit, and the relative density of serotypes
changes. In some cases the equilibrium between pneumococcus and the host
is destabilized resulting in rapid growth of pneumococcus which results
in clinical manifestations. It is thought that this is most likely to
occur immidietly following the acquisition of new serotype into the
nasopharyngeal flora. This most commonly occurs in the upper respiratory
tract where pneumococcus is generally located and results in the common
clinical manifestations of pneumococcal infections, i.e.~AOM, acute
sinusitis and conjunctivitis. The pathogenesis of pneumococcal pneumonia
is thought to occur through micro-aspiration of upper respiratory
secretions with subsequent rapid proliferation of pneumococcus in the
lower respiratory tract. Invasive disease occurs when pneumococcus
penetrates the host immunological defenses and proliferates in normally
sterile tissue. This can be secondary to infections of the upper or
lower respiratory tract, or can occur as a primary event. Generally, IPD
is considered to encompass meningitis, bacteraemia and septic arthritis.
While some may argue that the middle ear is normally sterile, AOM is not
considered invasive disease.

It has been known from the first pneumococcal vaccine trials that
vaccination has different efficacy against the different manifestations
of pneumococcal disease. The largest effects are consistently seen in
the prevention of IPD and pneumococcal lobar pneumonia. The effects on
carriage and AOM are often lesser in magnitude. This may be a true
biological gradient or a consequence of the accuracy with which disease
is measured. The more serious the illness is the more testing is
performed resulting in a more accurate diagnosis. Much of the AOM
attributed to pneumococcus may be caused by other pathogens, while by
definition IPD is always caused by pneumococcus. Furthermore, it wasn't
until the advent of pneumococcal conjugate vaccines that vaccines
started to become efficacious in children and it is precisely in
children that AOM occurs. The largest trials of modern pneumococcal
vaccines have fit the above narrative. The 23-valent pneumococcal
polysaccharide vaccine was trialed in 12,000 adults and showed an
efficacy of 75\% in preventing IPD and a 50\% efficacy against
radiologically confirmed pneumonia. The heptavalent pneumococcal
conjugate vaccine was trialed in children and produced a 97\% efficacy
in preventing IPD.

\subsection{Carriage}\label{carriage}

\textasciitilde{} 3-4 pages - Define carriage; age-specific prevalence,
serotype distribution - Explain that most are born carriage free -
Evidence for co-carriage of different serotypes - Age at which most
children acquire carriage - Risk factor: daycare, siblings, smoking etc.
- Children are the main vectors of pneumococcus - Rate of clearance
dependent on age - With increasing age -\textgreater{} increasing
immunity, decreasing prevalence - Senescence and carriage in the elderly
- Evidence for carriage being the predecessor infections - Evidence of
asymptomatic carriage -\textgreater{} main spread of disease - Variable
propensity of serotypes to cause disease, attack-rates - Review the
Icelandic literature and changing epidemiology - Carriage prevalence -
Serotype distribution - Risk factors

\subsection{Acute otitis media}\label{acute-otitis-media}

\textasciitilde{} 3 - 4 pages - Define different types of otitis media;
acute otitis media - Pathogens, estimated \% caused by pneumococcus -
Proposed mechanism by which carriage -\textgreater{} AOM - Epidemiology,
both serotype and age - Risk factors - Burden of disease caused by AOM;
health care utilization, cost - Incidence and prevalence - GP visits,
antibacterial consumption, hospitalization (?) - Days of work-lost by
parents - Sequelae; multiple infections, effusion, tympanostomies -
Evidence of benefit of delaying 1st presentation - Review of Icelandic
literature and changing epidemiology - AOM prevalence and serotype
distribution - Risk factors - Associated healthcare consumption, cost -
Rate of sequelae

\subsection{Pneumonia}\label{pneumonia}

\textasciitilde{} 4 -- 5 pages - Define: CAP, nosocomial, PP, NBPP and
IPP. - Pathogens, estimated \% caused by pneumococcus - Proposed
mechanism by which carriage -\textgreater{} pneumonia - Epidemiology,
both serotype and age - Risk factors - Burden of disease caused by
pneumonia, health care utilization - Ways of defining severity; CURB-65
etc. - GP visits, antibacterial consumption, Hospitalization rates -
Days of work lost - Mortality, sequelae - Review of Icelandic literature
and changing epidemiology - Pneumococcal pneumonia prevalence and
serotype distribution - Rate of hospitalization, healthcare consumption
- Rate of sequelae - Risk factors

\subsection{Invasive pneumococcal
disease}\label{invasive-pneumococcal-disease}

\textasciitilde{} 3 -5 pages - Define different presentations of IPD:
meningitis, bacteremia, etc. - Epidemiology, both serotype and age -
Risk factors - Burden of disease, health care utilization - Severity -
Hospitalization rates, ICU rates - Sequelae - Review of Icelandic
literature and changing epidemiology - Meningitis, bacteremia, empyema,
joint infection prevalence and serotype distribution - Rate of sequelae

\section{Pneumococcal conjugate
vaccines}\label{pneumococcal-conjugate-vaccines}

The history of pneumococcal vaccination begins in 1911 when Wright and
colleagues attempted to use whole killed bacteria to vaccinate South
African miners against pneumococcal pneumonia \citep{Wright1914}. It
should however be noted that in his original description of pneumococcus
in 1881, George Sternberg observed that rabbits who were injected with
his saliva mixed with alcohol and quinine did not ubiquitously die and
were later resistant to re-injection with saliva
\citep[\citet{Sternberg1881}]{Austrian1999a}. Sternberg had
inadvertently immunized the laboratory animals against subsequent
infection by injecting killed pneumococci, proving the concept 30 years
before it was first attempted. The 1911 trial by Wright failed to show
efficacy due to lack of knowledge of the significance of serotypes and
serotype specific immunogenicity. Several trials using whole killed
bacteria were published in the following two decades
\citep{Cecil1918, Lister1916, Lister1936, Maynard1913} The researchers
benefited from the knowledge of serotypes which had been discovered in
1910 and used multivalent vaccines. They were however victims of
underdeveloped epidemiological methodology for vaccine field trials. Due
to inconsistencies in study design, the efficacy of whole bacteria
pneumococcal vaccines remained fiercely debated though there seemed to
be a suggestion of benefit \citep{Austrian1999a}. Vaccines based on
whole killed bacteria were soon replaced with polysaccharide vaccines,
following discoveries in the 1920s and 1930 of the immunogenicity of the
polysaccharide capsule
\citep{Dochez1917, Finland1931, Francis1930, Heidelberger1923, Schiemann1927}.
The first such trial tested a bivalent polysaccharide vaccine on 29,000
adult males in the American Civilian Conservation Corps in iterations in
the 1930s \citep{Ekwurzel1938}. It suffered from the same methodological
problems as did the previous trials of the whole killed bacteria and its
results were debated. A second large trial was conducted in the late
1930s using a tetravalent vaccine \citep{Macleod1945}. This trial built
upon the experience of the previous trials and was able to show
convincing efficacy against pneumococcal pneumonia. The results of this
trial led to the licensure of two hexavalent polysaccharide pneumococcal
vaccines in the 1940s. One was formulated for adults and the other for
children, each optimized to the serotype distribution within the
respective age-group. Alas, these early vaccines fell victim to
unfortunate timing. Because in 1944, Tillet and colleagues showed that
bacteraemic pneumococcal pneumonia could be cured by parenteral
administration of benzylpenicillin \citep{Tillett1943}. Following this
discovery, the medical community was stricken with a kind of
nonchalance. The mortality rate of pneumococcal disease decreased
sufficiently that there was no longer a perceived need for preventative
vaccination. The licenses for the polysaccharide vaccines were withdrawn
by the manufacturer due to lack of their use \citep{Austrian1999a}
Interest in pneumococcal vaccination re-emerged in the 1950s when it was
noted that the mortality benefit of penicillin was not ubiquitous. The
elderly and those who had underlying disease did not experience a
decrease in their case fatality ratio \citep{Austrian1964}. This led to
a redoubled effort to create a new polysaccharide vaccine. Several large
randomized controlled trials were conducted in South Africa in the 1970s
\citep[\citet{Smit1977}]{Austrian1976} and on the basis of these, a
14-valent pneumococcal vaccine was licensed in the United States in
1977. Its valency was increased to 23 polysaccharides in 1983
\citep{Austrian1999a}. Early in the development of pneumococcal
vaccinations there was an interested in vaccinating children. Two trials
were conducted in the early 1980s which attempted to use polysaccharide
vaccines in young children. Neither showed benefit
\citep{Makela1981, Sloyer1981}. This is perhaps unsurprising in light of
previous trials. The first polysaccharide trial that was conducted in
children in 1937 failed to detect any immunological response
\citep{Davies1937}. Laboratory studies in the 1930s and 1940s revealed
that the reason for this lack of efficacy was due to the thymus
independent immunse response to purely sacharide antigens. These same
studies showed that this could be remedied by adding a protein adjuvant,
thus inducing a T-cell response. The strategy of protein conjugation saw
its first success in the development of the \emph{Haemophilus
influenzae} type b vaccine. Subsequently, several different pneumococcal
conjugate vaccines entered phase II and phase III clinical trials in the
late 1990s \citep{Austrian1999a}. The first of which to receive
licensure was the seven valent pneumococcal conjugate vaccine which was
licensed in 2000 in the United States. It included purified
polysaccharides for seven serotypes of pneumococcus (4, 9V, 14, 19F,
23F, 18C and 6B) conjugated to CRM197 (PCV7\textsubscript{CRM197}), a
nontoxic variant of the diptheria toxin. It was shown to be efficacious
for IPD, pneumococcal pneumonia and AOM in several randomized trials
\citep{Black2000, Black2002c, Eskola2001, Fireman2003, Kilpi2003}.
Higher valancy conjugated vaccines were developed and recieved licensure
in the new millennium based on the randomized trials conducted for the
heptavalent conjugated vaccine. They have however been shown to be
effective in several cluster randomized trials and observational
studies.

\section{Impact of pneumococcal conjugate
vaccines}\label{impact-of-pneumococcal-conjugate-vaccines}

\textasciitilde{} 3 pages - Present evidence of magnitude of effect on
VT carriage \textbf{The three Dagan studies mentioned in
\citep{Eskola2001}} - Serotype distribution vs.~carriage prevalence -
Serotype replacement - Herd-effect, i.e.~effect on carriage of adults
and non-vaccinated

From their inception, dozens of randomized controlled trials evaluating
the efficacy of different pneumococcal conjugate vaccines have been
performed.

\subsection{Acute otitis media}\label{acute-otitis-media-1}

Acute otitis media is still most often caused by \emph{Streptococcus
pneumoniae} and \emph{Haemophilus influenzae} despite changes in
otopathogens. Prevention of IPD in children and the associated morbidity
and mortality was the driving force in the development of pneumococcal
conjugate vaccines. However, the public most often associates them with
AOM. Most children experience AOM and the dramatic decrease in incidence
following pneumococcal conjugate vaccination is what families noticed.
Despite this, AOM is a difficult outcome for trialist. AOM exists on a
continuum. It does not have universally adhered to diagnostic criteria
and its signs and symptoms greatly overlap with those of other common
diseases. Because AOM is benign and most often self-limited, the
probability that a child with AOM is even seen by a physician varies
greatly with parental health seeking behavior. Even when AOM is
accurately diagnosed it is not possible to ascertain the causative
pathogen without invasive sampling, which is not warranted given the
benign nature of the disease. This precludes measuring the serotype
specific effect of vaccination for most studies - and more importantly,
it precludes measuring the effect on pneumococcal AOM. Thus any
estimation of an effect of pneumococcal vaccination will necessarily by
diluted by the subjectiveness of AOM diagnosis and the continued lack of
protection against other otopathogens.

\subsubsection{Randomised controlled
trials}\label{randomised-controlled-trials}

Despite these difficulties, AOM has been associated with pneumococcal
vaccination in children from the beginning. It was used as an outcome
measure in the earliest trials of the pneumococcal polysaccharide
vaccines \citep{Makela1981, Sloyer1981}. The first published randomized
controlled trial of a pneumococcal conjugate vaccine reported, among
other outcomes, the efficacy against AOM \citep{Black2000}. The study
recruited 37,868 children between October 1995 and August 1998 and
randomized them to the either PCV7\textsubscript{CRM197} or the
meningococcus C CRM197 conjugate vaccine. On the basis of a planned
interim analysis in August of 1998 the study met predefined efficacy
criteria and the Study Advisory Group recommended termination of the
trial. Blinded follow-up continued until April 20, 1999. However, for
the AOM portion of the paper, the data had only been analysed until
April 1998 A seperate publication from the same trial was published in
2003, and examined the effect of PCV7\textsubscript{CRM197} on AOM in
more detail using the full data until study completion in April 1999
\citep{Fireman2003}. Median follow-up time was not reported in either
publication, but 89\% children were reported to have completed the
primary series of vaccination in the Fireman et al paper. The data on
AOM was obtained from routine electronic health records. The assessors
were not specifically trained to evaluate AOM as these were simply
routine visits. The outcome measure AOM was defined in at least eight
different ways to account for the difficulties in measurement. Visits
and episodes were defined separately. A visit was considered to be due
to the same episode of AOM if the child presented within 21 days of a
previous AOM associated visit. Frequent otitis media was then defined as
either three episodes within a six month period, or four episodes within
a twelve month period. It is unclear exactly which statistical
procedures were used for which outcomes. Both the Andersen-Gill
extension of the Cox proportional hazards model with robust variance
estimation and the binomial test with Klopper-Pearson confidence
intervals were used and efficacy was reported as
\((1 - ratio\ measure) *100\%\). The study presented both per-protocol
and intention-to-treat estimates. Only the per-protocol effects will be
examined in this thesis though none of the intention to treat results
diverged from them. The estimated vaccine efficacy against otitis media
visits was 7.8\% (95\%CI 5.4\%-10.2\%). Slightly higher point estimates
were found for otitis media episodes, frequent otitis media and
ventilatory tube placements \citep{Black2000, Fireman2003}

The following year the results of another randomized controlled trial
were reported \citep{Eskola2001}. This study compared two heptavalent
pnuemococcal vaccines to a hepatitis B vaccine control. The two
heptavalent pneumococcal vaccines differed in their use of carrier
protein. One was the same vaccine as in the Black et al. study
(PCV7\textsubscript{CRM197}), and the other was a conjugated to
meningococcal outer membrane protein complex
(PCV7\textsubscript{MOMPC}). This publication only reported the
comparison of the PCV7\textsubscript{CRM197} to the hepatitis B vaccine.
The analogous comparison of the (PCV7\textsubscript{MOMPC}) was reported
in a seperate publication \citep{Kilpi2003}. No head-to-head comparison
of the two heptavalent vaccines was ever reported. The study methodology
was identical between the two publications as they report different arms
of the same study \citep{Eskola2001, Kilpi2003}. The study was
specifically designed to address the difficulties associated with
estimating the effect of pneumococcal vaccination on AOM. A total of
2,497 children were enrolled between December 1995 and April 1997, of
which 835 received the (PCV7\textsubscript{MOMPC}) vaccine and were
therefore not reported in the Eskola et al. paper. Children were
followed until their last visit at 24 months of age. Of the enrolled
children, 95.1\% completed full follow-up time and there was no evidence
of differential dropout. The study defined beforehand the criteria for
what constituted AOM and employed a trained study nurse and physician at
each study site. Children were seen at enrollment at two months of age,
and periodically assessed thereafter at four, six, seven, twelve,
thirteen and 24 months of age. Parents were encouraged to present with
their child to one of the study clinics for assessment of any symptoms
suggesting respiratory infection or AOM. If AOM was diagnosed as defined
by the study criteria, myringotomy and aspiration of middle-ear fluid
were performed and samples sent for culture. In this way, the study was
able to deduce the causative otopathogen. Episodes of AOM were
classified as all-cause AOM; culture-confirmed and otopathogen specific
AOM; and AOM due to serotypes included in the vaccine. The statistical
analysis was again conducted using the Andersen-Gill extension of the
Cox proportional hazards model with robust variance estimates and
efficacy was reported as \((1 - hazard\ ratio) *100\%\). The results
were most consistent with a 6\% efficacy against all-cause AOM with 95\%
confidence limits of -4\% and 16\%. In this case the negative lower
confidence limit indicates the data could be consistent with the
possibility of a 4\% increase in all-cause AOM, given the specified
model. The PCV\textsubscript{CRM197} efficacy against culture-confirmed
pneumococcal AOM was 35\% (95\%CI 21\%-45\%) and was 57\% (95\%CI
44\%-67\%) for the seven serotypes included in the vaccine. Similarly,
the study demonstrated 57\% (95\%CI 27\%-76\%) efficacy against AOM
caused by serotype 6A, which is considered a cross-reactive pneumococcal
serotype. The study was also one of the first to demonstrate clinically
relevant serotype replacement, showing a 33\% (95\%CI -1\%-80\%)
increase in pneumococcal AOM caused by serotypes not included in the
vaccine.

The effect estimates for the PCV7\textsubscript{MOMPC} against
culture-confirmed pneumococcal AOM was 25\% (95\%CI 11\%-37\%) and was
56\% (95\%CI 44\%-66\%) for the seven serotypes included in the vaccine.
However, unlike PCV\textsubscript{CRM197}, it did not seem to confer
protection against cross-reactive serotypes. Interestingly, virtually no
effect was seen on all-cause AOM with this vaccine preperation. The
effect estimate was -1\% (95\%CI -12\%-10\%). These suprising results
were not presented in the main text and no explanation given in the
discussion chapter \citep{Kilpi2003}.

\begin{tabular}{llllrl}
\toprule
Study & Vaccine & Enrollment period & Country & No. of children & Efficacy against Otitis media episodes\\
\midrule
Black, 2000 \& Fireman, 2003 & PCV7 (CRM197) & Oct 1995-Aug 1998 & United States & 37868 & 7.8\% (5.4\%-10.2\%)\\
Eskola, 2001 & PCV7 (CRM197) & Dec 1995-Apr 1997 & Finland & 1662 & 6\% (-4\%-16\%)\\
Kilpi, 2003 & PCV7 (MOMPC) & Dec 1995-Apr 1997 & Finland & 1666 & -1\% (-12\%-10\%\\
\bottomrule
\end{tabular}

\subsection{Pneumonia}\label{pneumonia-1}

\textasciitilde{} 2-3 pages - Present evidence of effect on all-cause
pneumonia - VT vs.~NVT serotypes - Serotype replacement (?) -
Herd-effect in adults and non-vaccinated

\subsection{Invasive pneumococcal
disease}\label{invasive-pneumococcal-disease-1}

\textasciitilde{} 4-6 pages \textless{}- largest amount of studies -
Present evidence of effect on IPD and subgroups; meningitis, bacteremia
etc. - VT vs.~NVT - Serotype replacement - Herd-effect

\section{Cost-effectiveness of pneumococcal conjugate
vaccination}\label{cost-effectiveness-of-pneumococcal-conjugate-vaccination}

\textasciitilde{} 3-4 pages - Present overview of literature review and
critical analysis. - Recommendations of ISPOR and WHO presented, discuss
importance of assumptions and methodology - Introduction to sub-chapters
of lit. rev. - Explain how they will be tied in to ISPOR/WHO
recommendations

\subsection{Measurement of effectiveness and choice of health
outcomes}\label{measurement-of-effectiveness-and-choice-of-health-outcomes}

\textasciitilde{} 1 page - Shortly explain what is meant by
effectiveness and health outcomes - Tie in to ISPOR/WHO

\subsubsection{Health outcomes
considered}\label{health-outcomes-considered}

\textasciitilde{} 2-3 pages - Describe what health outcomes were
considered - Tie into actual disease burden known to be caused by
pneumococcus

\subsubsection{Effectiveness of PCV7}\label{effectiveness-of-pcv7}

\textasciitilde{} 3-4 pages - What effectiveness rationale is used,
methods and rationale: critique - Carriage - AOM - Pneumonia - IPD

\subsubsection{Effectiveness of PCV10}\label{effectiveness-of-pcv10}

\textasciitilde{}2-3 pages - What effectiveness rationale is used,
methods and rationale: critique - Carriage - AOM - Pneumonia - IPD

\subsubsection{Effectiveness of PCV13}\label{effectiveness-of-pcv13}

\textasciitilde{} 2- 3 pages - What effectiveness rationale is used,
methods and rationale: critique - Carriage - AOM - Pneumonia - IPD

\subsection{Estimating resources and
cost}\label{estimating-resources-and-cost}

\textasciitilde{}1 page - Shortly explain what resources and costs mean
- Direct vs.~indirect - Tie in to ISPOR/WHO

\chapter{Aims}\label{aims}

\chapter{Materials and methods}\label{methods}

We describe our methods in this chapter.

\chapter{Results}\label{results}

\chapter{Discussion}\label{discussion}

\bibliography{library.bib,packages.bib}


\end{document}
