\documentclass[]{book}
\usepackage{lmodern}
\usepackage{amssymb,amsmath}
\usepackage{ifxetex,ifluatex}
\usepackage{fixltx2e} % provides \textsubscript
\ifnum 0\ifxetex 1\fi\ifluatex 1\fi=0 % if pdftex
  \usepackage[T1]{fontenc}
  \usepackage[utf8]{inputenc}
\else % if luatex or xelatex
  \ifxetex
    \usepackage{mathspec}
  \else
    \usepackage{fontspec}
  \fi
  \defaultfontfeatures{Ligatures=TeX,Scale=MatchLowercase}
\fi
% use upquote if available, for straight quotes in verbatim environments
\IfFileExists{upquote.sty}{\usepackage{upquote}}{}
% use microtype if available
\IfFileExists{microtype.sty}{%
\usepackage{microtype}
\UseMicrotypeSet[protrusion]{basicmath} % disable protrusion for tt fonts
}{}
\usepackage[margin=1in]{geometry}
\usepackage{hyperref}
\hypersetup{unicode=true,
            pdftitle={PhD thesis},
            pdfauthor={Elías Sæbjörn Eyþórsson},
            pdfborder={0 0 0},
            breaklinks=true}
\urlstyle{same}  % don't use monospace font for urls
\usepackage{natbib}
\bibliographystyle{apalike}
\usepackage{longtable,booktabs}
\usepackage{graphicx,grffile}
\makeatletter
\def\maxwidth{\ifdim\Gin@nat@width>\linewidth\linewidth\else\Gin@nat@width\fi}
\def\maxheight{\ifdim\Gin@nat@height>\textheight\textheight\else\Gin@nat@height\fi}
\makeatother
% Scale images if necessary, so that they will not overflow the page
% margins by default, and it is still possible to overwrite the defaults
% using explicit options in \includegraphics[width, height, ...]{}
\setkeys{Gin}{width=\maxwidth,height=\maxheight,keepaspectratio}
\IfFileExists{parskip.sty}{%
\usepackage{parskip}
}{% else
\setlength{\parindent}{0pt}
\setlength{\parskip}{6pt plus 2pt minus 1pt}
}
\setlength{\emergencystretch}{3em}  % prevent overfull lines
\providecommand{\tightlist}{%
  \setlength{\itemsep}{0pt}\setlength{\parskip}{0pt}}
\setcounter{secnumdepth}{5}
% Redefines (sub)paragraphs to behave more like sections
\ifx\paragraph\undefined\else
\let\oldparagraph\paragraph
\renewcommand{\paragraph}[1]{\oldparagraph{#1}\mbox{}}
\fi
\ifx\subparagraph\undefined\else
\let\oldsubparagraph\subparagraph
\renewcommand{\subparagraph}[1]{\oldsubparagraph{#1}\mbox{}}
\fi

%%% Use protect on footnotes to avoid problems with footnotes in titles
\let\rmarkdownfootnote\footnote%
\def\footnote{\protect\rmarkdownfootnote}

%%% Change title format to be more compact
\usepackage{titling}

% Create subtitle command for use in maketitle
\newcommand{\subtitle}[1]{
  \posttitle{
    \begin{center}\large#1\end{center}
    }
}

\setlength{\droptitle}{-2em}
  \title{PhD thesis}
  \pretitle{\vspace{\droptitle}\centering\huge}
  \posttitle{\par}
  \author{Elías Sæbjörn Eyþórsson}
  \preauthor{\centering\large\emph}
  \postauthor{\par}
  \predate{\centering\large\emph}
  \postdate{\par}
  \date{2018-09-23}

\usepackage{booktabs}
\usepackage{amsthm}
\makeatletter
\def\thm@space@setup{%
  \thm@preskip=8pt plus 2pt minus 4pt
  \thm@postskip=\thm@preskip
}
\makeatother

\usepackage{amsthm}
\newtheorem{theorem}{Theorem}[chapter]
\newtheorem{lemma}{Lemma}[chapter]
\theoremstyle{definition}
\newtheorem{definition}{Definition}[chapter]
\newtheorem{corollary}{Corollary}[chapter]
\newtheorem{proposition}{Proposition}[chapter]
\theoremstyle{definition}
\newtheorem{example}{Example}[chapter]
\theoremstyle{definition}
\newtheorem{exercise}{Exercise}[chapter]
\theoremstyle{remark}
\newtheorem*{remark}{Remark}
\newtheorem*{solution}{Solution}
\begin{document}
\maketitle

{
\setcounter{tocdepth}{1}
\tableofcontents
}
\chapter{Preamble}\label{preamble}

I am currently writing my PhD thesis on the impact of pneumococcal
vaccination in Iceland. I decided to host the thesis on github and
distribute on social media. I am doing this for mostly selfish reasons.
I believe I will be more motivated if my productiveness -- or lack
thereof, is held accountable to anyone who wishes to check. I would be
grateful for any and all comments on any aspect of the thesis under
construction.

\chapter{Introduction}\label{intro}

\emph{Streptococcus pneumoniae} is a commensal bacterium found in the
nasopharynx of humans where it plays an integral role in the normal
upper respiratory flora. It is also a common pathogen and is one of the
most common bacterial causes of disease in humans. In classical medical
texts pneumococcus is described as a Gram-positive lancet-shaped coccus
that is usually found in pairs. In fact, pneumococcus is \emph{the}
Gram-positive coccus, in that it was the first bacteria that Christian
Gram noted to retain the dark aniline-gentian violet stain that now
bears his name \citep{Gram1884}. Pneumococcus was first isolated in 1881
by two microbiologist, George M. Sternberg in the United States and
Loius Pasteur in France \citep{Pasteur1881, Sternberg1881, Watson1993}.
In both cases the sample from which the bacterium was isolated
originated from healthy carriers, rather than ill patients. The causal
association between this newly discovered bacterium and pneumonia was
firmly established only five years later \citep{Weichselbaum1886}, and
in the following decade, all clinical presentations of pneumococcal
infection had been described \citep{Austrian1981}. Pneumococcus has gone
by many names from its first isolation in 1881. Originally named
\emph{Micrococcus pasteuri} by Sternberg \citep{Sternberg1881}, a
scientific consensus was reached in 1920 whereupon it was agreed that
the official name should be \emph{Diplococcus pneumoniae}
\citep{Winslow1920}. It was not until 1974 when pneumococcus received
its current name, \emph{Streptococcus pneumoniae} \citep{Deibel1974}.

Pneumococcus is encapsulated by a polysaccharide coating that protects
it from environmental factors. The polysaccharide capsule acts as an
invisibility cloak in regard to the human immune system which is
rendered unable to detect pneumococcus except through certain patterns
in the oligosaccharides contained within the capsule
\citep{Epstein1995}. Based on these patterns, pneumococcus has been
classified into over 97 different serotypes as of the writing of this
thesis. Additionally, as the capsule only contains polysaccharides and
not proteins, the immune response is T cell independent and therefore
poorly immunogenic even after identification by the immune system
\citep{Geno2015b}. These are the fundamental problems faced by
scientists when engineering new pneumococcal vaccines. The significance
of serotypes on the development of vaccines should not be understated.
Lack of knowledge about serotype-specific immunogenicity was the main
reason for the failure of the original attempts at preventing
pneumococcal pneumonia by Wright in South Africa \citep{Wright1914}.
Because of this, the feasibility of pneumococcal vaccination wasn't
definitively demonstrated until 1945 \citep{Macleod1945}.

All serotypes of pneumococcus have the potential to cause disease in
humans. However, some are more virulent than others. The prevalence of
asymptomatic carriage in the nasopharynx varies greatly by serotype as
does the prevalence of serotypes causing clinical infections.
Quantifying the pathogenic potential of serotypes can be difficult as
both need to be considered. With few exemptions, acquisition of a new
serotype into the nasopharyngeal flora proceeds clinical disease caused
by that serotype. Pneumococcal epidemiology is dominated by this effect,
wherein children act as a reservoir of asymptomatic pneumococcal
carriage from which other children and adults acquire serotypes that may
lead to symptomatic disease. Vaccinations that decrease the pneumococcal
carriage in children do therefore have the potential to reduce the
incidence of disease in both other unvaccinated children and adults.
This phenomenon is called a herd-effect and is integral to the
development of vaccination strategies to combat pneumococcal disease.

The infectious manifestations of pneumococcal disease are broadly
speaking local infections of the respiratory tract and infections of
previously sterile sites. They range from common to uncommon and from
benign to serious. The most common infectious manifestation of
pneumococcus in acute otitis media (AOM) -- an infection of the middle
ear. AOM is the most common reason for physician visit and antimicrobial
prescription in the pediatric population. However, the disease course is
benign and rarely results in permanent disability. The pathogenesis of
AOM is complex and can both be caused by viral and bacterial pathogens.
Bacterial AOM is most often caused by pneumococcus and Haemophilus
influnzae. Acute sinusitis is another common but benign manifestation of
pneumococcal disease. Pneumococcus is also a common cause of pneumonia
-- from which it gets its name. Pneumonia often requires hospitalization
and intravenous antimicrobial treatment and can uncommonly lead to
permanent disability and death. Finally, if pneumococcus gains access to
normally sterile sites it may cause invasive infections. This includes
bacteremia - an infection of the blood, and meningitis -- an infection
of the meninges. These infectious manifestations are grouped together as
invasive pneumococcal disease (IPD). Whilst IPD is extremely uncommon
the consequences can be disastrous. The case fatality ratio from
pneumococcal meningitis in Iceland is estimated at 15.3\%.

Pneumococcus became an early target for vaccine development because of
this broad range of disease caused by pneumococcus. The vaccine benefit
can be quantified in two different ways. On one hand it can prevent
uncommon but serious manifestations of disease and has the potential to
prevent death and serious disability. On the other hand, it prevents
common infections that cumulatively present a large healthcare burden
due to frequent physician visits, antimicrobial prescriptions and days
of work lost by parents and caregivers. The earliest attempts to use
vaccination to lessen the morbidity and mortality associated with
pneumococcus date back to 1914 when Wright used whole cell innoculi to
vaccinated miners in South Africa \citep{Wright1914}. In 1934, armed
with the knowledge of different pneumococcal serotypes and serotype
specific immunogenicity, Lister and Ordman demonstrated the
effectiveness of polyvalent whole-cell vaccines against pneumococcal
pneumonia. In the following decades, repeated animal studies showed that
injection of the pneumococcal polysaccharide coating in animals
protected against subsequent pneumococcal infections. On the basis of
these findings, the first polysaccharide vaccine was demonstrated to be
effective \citep{Macleod1945}. This lay the foundation for modern
polysaccharide pneumococcal vaccines. Unfortuneatly, it soon became
apparent that the polysaccharide vaccines were not addequetly
immunogenic in young children, the ill and the elderly. This lead to the
development of pneumococcal conjugate vaccines. The first such vaccine
to be mass produced was the heptavalent Prevenar developed by Pfizer
pharmacouticals.

Iceland is an independent island nation located in the mid-Atlantic. It
has a homogenous populuation of roughly 330,000 individuals. The first
systamatic vaccination against pneumococcus in Iceland began in April
2011 when the 10-valent pneumococcal \emph{Haemophilus influnzae}
protein-D conjugate vaccine (Synflorix, PHiD-CV10) was introduced into
the national Paediatric vaccination program. The vaccine program
entailed two primary doses at three and five months of age and a booster
dose at twelve months. No catch-up program was undertaken. Prior to the
introduction, no systamatic vaccination effort had been undertaken in
Iceland. Like the other Nordic countries, Iceland has a rich history of
national health related registers. This allowed for a unique whole
population ecological study to examine the impact of systamatic
pneumococcal vaccination.

\section{\texorpdfstring{Clinical manifestations of \emph{Streptococcus
pneumoniae}}{Clinical manifestations of Streptococcus pneumoniae}}\label{clinical-manifestations-of-streptococcus-pneumoniae}

\textasciitilde{} 2 pages - Build transmission dynamics, carriage and
disease mentioned in introduction - Short overview of mechanism by which
individuals become colonized - Asymptomatic carriage the predecessor of
infections - Non-invasive vs.~invasive infections - Explain relevance of
differentiating the two - list manifestations in each category building
on introduction - Explain that vaccines have variable impact on
different manifestations - Use examples from the two large RCTs

The relationship between pneumococcus and humans is complex. Most
children are colonized by pneumococcus within the first months of life.
The serotype distribution of the initial colonization is influenced by
the distribution of serotypes within the family. Over the course of the
child's lifetime, they will be colonized by many different serotypes.
The child's immune system will learn to recognize the serotype and will
either clear the colonization or maintain a equilibrium in which the
serotype is kept within a certain limit of reproduction. Consequently,
the contribution of pneumococcus to the human upper respiratory flora is
in a state of constant flux. New serotypes enter while old exit, and the
relative density of serotypes changes. In some cases the equilibrium
between pneumococcus and the host is destabilized resulting in rapid
growth of pneumococcus which results in clinical manifestations. This
most commonly occurs in the upper respiratory tract where pneumococcus
is generally located. This results in the common clinical manifestations
of pneumococcal infections, i.e.~AOM, acute sinusitis and
conjunctivitis. The pathogenesis of pneumococcal pneumonia is thought to
occur through micro-aspiration of upper respiratory secretions with
subsequent rapid proliferation of pneumococcus in the lower respiratory
tract. Invasive disease occurs when pneumococcus penetrates the host
immunological defenses and proliferates in normally sterile tissue.

\subsection{Carriage}\label{carriage}

\textasciitilde{} 3-4 pages - Define carriage; age-specific prevalence,
serotype distribution - Explain that most are born carriage free -
Evidence for co-carriage of different serotypes - Age at which most
children acquire carriage - Risk factor: daycare, siblings, smoking etc.
- Children are the main vectors of pneumococcus - Rate of clearance
dependent on age - With increasing age -\textgreater{} increasing
immunity, decreasing prevalence - Senescence and carriage in the elderly
- Evidence for carriage being the predecessor infections - Evidence of
asymptomatic carriage -\textgreater{} main spread of disease - Variable
propensity of serotypes to cause disease, attack-rates - Review the
Icelandic literature and changing epidemiology - Carriage prevalence -
Serotype distribution - Risk factors - Describe mathematical method to
estimate adult carriage from observed carriage in children.

\subsection{Acute otitis media}\label{acute-otitis-media}

\textasciitilde{} 3 - 4 pages - Define different types of otitis media;
acute otitis media - Pathogens, estimated \% caused by pneumococcus -
Proposed mechanism by which carriage -\textgreater{} AOM - Epidemiology,
both serotype and age - Risk factors - Burden of disease caused by AOM;
health care utilization, cost - Incidence and prevalence - GP visits,
antibacterial consumption, hospitalization (?) - Days of work-lost by
parents - Sequelae; multiple infections, effusion, tympanostomies -
Evidence of benefit of delaying 1st presentation - Review of Icelandic
literature and changing epidemiology - AOM prevalence and serotype
distribution - Risk factors - Associated healthcare consumption, cost -
Rate of sequelae

\subsection{Pneumonia}\label{pneumonia}

\textasciitilde{} 4 -- 5 pages - Define: CAP, nosocomial, PP, NBPP and
IPP. - Pathogens, estimated \% caused by pneumococcus - Proposed
mechanism by which carriage -\textgreater{} pneumonia - Epidemiology,
both serotype and age - Risk factors - Burden of disease caused by
pneumonia, health care utilization - Ways of defining severity; CURB-65
etc. - GP visits, antibacterial consumption, Hospitalization rates -
Days of work lost - Mortality, sequelae - Review of Icelandic literature
and changing epidemiology - Pneumococcal pneumonia prevalence and
serotype distribution - Rate of hospitalization, healthcare consumption
- Rate of sequelae - Risk factors

\subsection{Invasive pneumococcal
disease}\label{invasive-pneumococcal-disease}

\textasciitilde{} 3 -5 pages - Define different presentations of IPD:
meningitis, bacteremia, etc. - Epidemiology, both serotype and age -
Risk factors - Burden of disease, health care utilization - Severity -
Hospitalization rates, ICU rates - Sequelae - Review of Icelandic
literature and changing epidemiology - Meningitis, bacteremia, empyema,
joint infection prevalence and serotype distribution - Rate of sequelae

\section{Pneumococcal conjugate
vaccines}\label{pneumococcal-conjugate-vaccines}

\textasciitilde{} 1 -2 pages - PPSV23 original studies, downsides,
immunogenicity - Development of protein conjugate vaccines, reasons -
PCV7 - Higher valency PCVs

\section{Impact of pneumococcal conjugate
vaccines}\label{impact-of-pneumococcal-conjugate-vaccines}

\textasciitilde{} 3 pages - Present evidence of magnitude of effect on
VT carriage - Serotype distribution vs.~carriage prevalence - Serotype
replacement - Herd-effect, i.e.~effect on carriage of adults and
non-vaccinated

\subsection{Acute otitis media}\label{acute-otitis-media-1}

\textasciitilde{} 1-2 pages \textless{}- much fewer studies - Present
evidence of magnitude of effect on all-cause AOM - VT vs.~NVT serotypes
- Serotype replacement (?)\\
- Herd-effect in non-vaccinated

\subsection{Pneumonia}\label{pneumonia-1}

\textasciitilde{} 2-3 pages - Present evidence of effect on all-cause
pneumonia - VT vs.~NVT serotypes - Serotype replacement (?) -
Herd-effect in adults and non-vaccinated

\subsection{Invasive pneumococcal
disease}\label{invasive-pneumococcal-disease-1}

\textasciitilde{} 4-6 pages \textless{}- largest amount of studies -
Present evidence of effect on IPD and subgroups; meningitis, bacteremia
etc. - VT vs.~NVT - Serotype replacement - Herd-effect

\section{Cost-effectiveness of pneumococcal conjugate
vaccination}\label{cost-effectiveness-of-pneumococcal-conjugate-vaccination}

\textasciitilde{} 3-4 pages - Present overview of literature review and
critical analysis. - Recommendations of ISPOR and WHO presented, discuss
importance of assumptions and methodology - Introduction to sub-chapters
of lit. rev. - Explain how they will be tied in to ISPOR/WHO
recommendations

\subsection{Measurement of effectiveness and choice of health
outcomes}\label{measurement-of-effectiveness-and-choice-of-health-outcomes}

\textasciitilde{} 1 page - Shortly explain what is meant by
effectiveness and health outcomes - Tie in to ISPOR/WHO

\subsubsection{Health outcomes
considered}\label{health-outcomes-considered}

\textasciitilde{} 2-3 pages - Describe what health outcomes were
considered - Tie into actual disease burden known to be caused by
pneumococcus

\subsubsection{Effectiveness of PCV7}\label{effectiveness-of-pcv7}

\textasciitilde{} 3-4 pages - What effectiveness rationale is used,
methods and rationale: critique - Carriage - AOM - Pneumonia - IPD

\subsubsection{Effectiveness of PCV10}\label{effectiveness-of-pcv10}

\textasciitilde{}2-3 pages - What effectiveness rationale is used,
methods and rationale: critique - Carriage - AOM - Pneumonia - IPD

\subsubsection{Effectiveness of PCV13}\label{effectiveness-of-pcv13}

\textasciitilde{} 2- 3 pages - What effectiveness rationale is used,
methods and rationale: critique - Carriage - AOM - Pneumonia - IPD

\subsection{Estimating resources and
cost}\label{estimating-resources-and-cost}

\textasciitilde{}1 page - Shortly explain what resources and costs mean
- Direct vs.~indirect - Tie in to ISPOR/WHO

\chapter{Aims}\label{aims}

\chapter{Materials and methods}\label{methods}

We describe our methods in this chapter.

\chapter{Results}\label{results}

\chapter{Discussion}\label{discussion}

\bibliography{library.bib,packages.bib}


\end{document}
